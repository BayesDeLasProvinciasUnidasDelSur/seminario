\documentclass[a4paper,10pt]{article}
\usepackage[utf8]{inputenc}
\input{../tex/encabezado.tex}
\input{../tex/tikzlibrarybayesnet.code.tex}

\usepackage{paracol}
%opening
\title{Clase 1}
\author{Gustavo Landfried}

\begin{document}

\maketitle
% 
% \begin{abstract}
%  Este documento es un complemento de la charla ``Técnicas para calcular distribuciones de creencias honestas'' y tiene por objetivo servir de herramienta para introducirse en inferencia Bayesiana mediante problemas t\'ipicos, de dificultad variable, en sus aspectos pr\'acticos y te\'oricos. Si bien son varios ejercicios en la consigna, la idea es que hagan uno y el resto les quede.
% \end{abstract}


\section*{Elije tu propia aventura}

Elija un solo ejercicio ($a$ o $b$), y resu\'elvalo.

\begin{enumerate}
 \item \textbf{Flujo de inferencia en modelos generativos} (Secci\'on \ref{sec:flujo})
 \begin{enumerate}
  \item \textbf{Implementar} un modelo gr\'afico utilizando el software \texttt{samiam} y determinar cuándo un flujo de inferencia permanece abierto (Subsecci\'on \ref{subsec:flujo})
  \begin{enumerate}[i]
  \item Definir los factores del modelo gr\'afico visto en clase
  \item Extender el modelo gr\'afico
  \item Observar el efecto de los observables sobre las creencias a priori y a posteriori
  \item Determinar cu\'ando un flujo de inferencia permanece abierto
 \end{enumerate}
  \item \textbf{Derivar} algunas de las creencias a priori y creencias a posteriori mediante las reglas de la probabilidad (Subsecci\'on \ref{ssec:reglas})
 \end{enumerate}
 \item \textbf{Habilidad en la industria del video juego} (Secci\'on \ref{sec:trueskill})
 \begin{enumerate}
  \item \textbf{Implementar} una estimaci\'on de habilidad sobre jugadores en el lenguaje de programci\'on de su preferencia (Subsecci\'on \ref{ssec:estimacion})
  \begin{enumerate}[i]
  \item Utilizar la librer\'ia TrueSkill de python o la respectiva a su lenguaje de programaci\'on
  \item Simular un jugador con oponentes al azar utilizando el modelo generativo
  \item Estimar la habilidad del jugador, considerando conocidos a los opnentes, $\sigma=1$
  \item Repetir el punto ii y iii, y observar como var\'ian las observaciones 
 \end{enumerate}
  \item \textbf{Derivar} la verosimilitud, la evidencia y la posterior del modelo TrueSkill usando el sum-product algorithm sobre su factor graph (Subsecci\'on \ref{ssec:sum_product})
 \end{enumerate} 
 \item \textbf{Regresi\'on lineal Bayesiana} (Secci\'on \ref{sec:lineal})
 \begin{enumerate}
 \item \textbf{Implementar} una selección de modelo Bayesiana sobre datos simulados en el lenguaje de programaci\'on de su preferencia (Subsecci\'on \ref{ssec:seleccion})
 \begin{enumerate}[i]
  \item Definir la posterior, la verosimilitud y la evidencia de la regresi\'on lineal
  \item Simular datos con ruido provenientes de una sinoidal
  \item Ajustar regresiones polinomiales de grado 0 hasta 9 a los datos
  \item Seleccionar modelo basado en la evidencia
 \end{enumerate}
 \item \textbf{Derivar} la distribuci\'on de creencias a posteriori y la evidencia de la regresi\'on lineal Bayesiana (Subsecci\'on \ref{ssec:gaussiana})
\end{enumerate}
\end{enumerate}

\newpage

\input{clase1/clase1.tex}

\end{document}
