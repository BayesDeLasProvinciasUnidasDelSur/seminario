\section{Anexo}

\subsection{M\'axima verosimilitud}

\begin{equation}\label{eq:maximum_likelihood}
 \begin{split}
   \text{log } p(\bm{t} | \bm{x}, \bm{w}, \beta) & = \sum_{i=1}^{n} \text{log } N(t_i | \bm{w}^T \bm{\phi}(\bm{x}_i), \sigma)  \\
  & =  \sum_{i=1}^{n} \text{log }  \frac{\sqrt{\beta} }{\sqrt{2\pi}} e^{\frac{-(t_i - \bm{w}^T\bm{\phi}(\bm{x}_i))^2}{2\beta^{-1}} } = \sum_{i=1}^{n} \text{log } \frac{\sqrt{\beta} }{\sqrt{2\pi}} + \sum_{i=1}^{n} \text{log } e^{\frac{-(t_i - \bm{w}^T\bm{\phi}(\bm{x}_i))^2}{2\beta^{-1}} } \\
  & = n \text{log } \frac{\sqrt{\beta} }{\sqrt{2\pi}} + \sum_{i=1}^{n} \text{log } e^{\frac{-(t_i - \bm{w}^T\bm{\phi}(\bm{x}_i))^2}{2\beta^{-1}} } = n \text{log } \frac{\sqrt{\beta} }{\sqrt{2\pi}} + \sum_{i=1}^{n}  \frac{-(t_i - \bm{w}^T\bm{\phi}(\bm{x}_i))^2}{2\beta^{-1}} \\
   &  = n \text{ log } \sqrt{\beta} - n \text{ log } \sqrt{2\pi} - \frac{\beta}{2} \sum_{i=1}^{n}  (t_i - \bm{w}^T\bm{\phi}(\bm{x}_i))^2   \\
  & \propto  - \sum_{i=1}^{n}  (t_i - \bm{w}^T\bm{\phi}(\bm{x}_i))^2 
 \end{split}
\end{equation}

\subsection{Multiplicaci\'on de normales}\label{multiplicacion_normales}

Luego, el problema que tenemos que resolver es
\begin{equation}
 \int N(x;\mu_1,\sigma_1^2)N(x;\mu_2,\sigma_2^2) dx
\end{equation}

Por defnici\'on,
\begin{equation}
\begin{split}
 N(x;y,\beta^2)N(x;\mu,\sigma^2) & = \frac{1}{\sqrt{2\pi}\sigma_1}e^{-\frac{(x-\mu_1)^2}{2\sigma_1^2}} \frac{1}{\sqrt{2\pi}\sigma_2}e^{-\frac{(x-\mu_2)^2}{2\sigma_2^2}}  \\
 & = \frac{1}{2\pi\sigma_1\sigma_2}\text{exp}\Bigg(-\underbrace{\left( \frac{(x-\mu_1)^2}{2\sigma_1^2} + \frac{(x-\mu_2)^2}{2\sigma_2^2} \right)}_{\theta} \Bigg)
\end{split}
\end{equation}

Luego,
\begin{equation}
 \theta = \frac{\sigma_2^2(x^2 + \mu_1^2 - 2x\mu_1) + \sigma_1^2(x^2 + \mu_2^2 - 2x\mu_2) }{2\sigma_1^2\sigma_2^2}
\end{equation}

Expando y reordeno los factores por potencias de $x$
\begin{equation}
 \frac{(\sigma_1^2 + \sigma_2^2) x^2 - (2\mu_1\sigma_2^2 + 2\mu_2\sigma_1^2) x + (\mu_1^2\sigma_2^2 + \mu_2^2\sigma_1^2)}{2\sigma_1^2\sigma_2^2}
\end{equation}

Divido al numerador y el denominador por el factor de $x^2$
\begin{equation}
 \frac{x^2 - 2\frac{(\mu_1\sigma_2^2 + \mu_2\sigma_1^2)}{(\sigma_1^2 + \sigma_2^2) } x + \frac{(\mu_1^2\sigma_2^2 + \mu_2^2\sigma_1^2)}{(\sigma_1^2 + \sigma_2^2) }}{2\frac{\sigma_1^2\sigma_2^2}{(\sigma_1^2 + \sigma_2^2)}}
\end{equation}

Esta ecuaci\'on es cuadr\'atica en x, y por lo tanto es proporcional a una funci\'on de densidad gausiana con desv\'io
\begin{equation}
\sigma_{\times} = \sqrt{\frac{\sigma_1^2\sigma_2^2}{\sigma_1^2+\sigma_2^2}}  
\end{equation}

y media
\begin{equation}
 \mu_{\times} = \frac{(\mu_1\sigma_2^2 + \mu_2\sigma_1^2)}{(\sigma_1^2 + \sigma_2^2) }
\end{equation}

Dado que un t\'ermino $\varepsilon = 0$ puede ser agregado para completar el cuadrado en $\theta$, esta prueba es suficiente cuando no se necesita una normalizaci\'on.
Sea, 
\begin{equation}
 \varepsilon = \frac{\mu_{\times}^2-\mu_{\times}^2}{2\sigma_{\times}^2} = 0
\end{equation}

Al agregar este t\'ermino a $\theta$ tenemos
\begin{equation}
 \theta = \frac{x^2 - 2\mu_{\times}x + \mu_{\times}^2 }{2\sigma_{\times}^2} + \underbrace{\frac{ \frac{(\mu_1^2\sigma_2^2 + \mu_2^2\sigma_1^2)}{(\sigma_1^2 + \sigma_2^2) } - \mu_{\times}^2}{2\sigma_{\times}^2}}_{\varphi}
\end{equation}

Reorganizando el t\'ermino $\varphi$
\begin{equation}
\begin{split}
\varphi & = \frac{\frac{(\mu_1^2\sigma_2^2 + \mu_2^2\sigma_1^2)}{(\sigma_1^2 + \sigma_2^2) } - \left(\frac{(\mu_1\sigma_2^2 + \mu_2\sigma_1^2)}{(\sigma_1^2 + \sigma_2^2) }\right)^2 }{2\frac{\sigma_1^2\sigma_2^2}{\sigma_1^2+\sigma_2^2}}  \\
& = \frac{(\sigma_1^2 + \sigma_2^2)(\mu_1^2\sigma_2^2 + \mu_2^2\sigma_1^2) - (\mu_1\sigma_2^2 + \mu_2\sigma_1^2)^2}{\sigma_1^2 + \sigma_2^2}\frac{1}{2\sigma_1^2\sigma_2^2} \\[0.3cm]
& = \frac{(\mu_1^2\sigma_1^2\sigma_2^2 + \cancel{\mu_2^2\sigma_1^4} + \bcancel{\mu_1^2\sigma_2^4} + \mu_2^2\sigma_1^2\sigma_2^2) - (\bcancel{\mu_1^2\sigma_2^4} + 2\mu_1\mu_2\sigma_1^2\sigma_2^2 + \cancel{\mu_2^2\sigma_1^4} )}{\sigma_1^2 + \sigma_2^2}  \frac{1}{2\sigma_1^2\sigma_2^2} \\[0.3cm] 
& = \frac{(\sigma_1^2\sigma_2^2)(\mu_1^2 + \mu_2^2 - 2\mu_1\mu_2)}{\sigma_1^2 + \sigma_2^2}\frac{1}{2\sigma_1^2\sigma_2^2} = \frac{\mu_1^2 + \mu_2^2 - 2\mu_1\mu_2}{2(\sigma_1^2 + \sigma_2^2)} = \frac{(\mu_1 - \mu_2)^2}{2(\sigma_1^2 + \sigma_2^2)}
\end{split}
\end{equation}

Luego,
\begin{equation}
 \theta = \frac{(x-\mu_{\times})^2}{2\sigma_{\times}^2} + \frac{(\mu_1 - \mu_2)^2}{2(\sigma_1^2 + \sigma_2^2)} 
\end{equation}

Colocando esta forma de $\theta$ en su lugar
\begin{equation}
\begin{split}
 N(x;y,\beta^2)N(x;\mu,\sigma^2) & = \frac{1}{2\pi\sigma_1\sigma_2}\text{exp}\Bigg(-\underbrace{\left( \frac{(x-\mu_{\times})^2}{2\sigma_{\times}^2} + \frac{(\mu_1 - \mu_2)^2}{2(\sigma_1^2 + \sigma_2^2)} \right)}_{\theta} \Bigg) \\
 & = \frac{1}{2\pi\sigma_1\sigma_2}\text{exp}\left(  - \frac{(x-\mu_{\times})^2}{2\sigma_{\times}^2} \right) \text{exp} \left( - \frac{(\mu_1 - \mu_2)^2}{2(\sigma_1^2 + \sigma_2^2)} \right) 
\end{split}
\end{equation}

Multiplicando por $\sigma_{\times}\sigma_{\times}^{-1}$
\begin{equation}
\overbrace{\frac{\cancel{\sigma_1\sigma_2}}{\sqrt{\sigma_1^2+\sigma_2^2}}}^{\sigma_{\times}} \frac{1}{\sigma_{\times}} \frac{1}{2\pi\cancel{\sigma_1\sigma_2}}\text{exp}\left(  - \frac{(x-\mu_{\times})^2}{2\sigma_{\times}^2} \right) \text{exp} \left( - \frac{(\mu_1 - \mu_2)^2}{2(\sigma_1^2 + \sigma_2^2)} \right)
\end{equation}

Luego,
\begin{equation}
 \frac{1}{\sqrt{2\pi}\sigma_{\times}}\text{exp}\left(  - \frac{(x-\mu_{\times})^2}{2\sigma_{\times}^2} \right) \frac{1}{\sqrt{2\pi(\sigma_1^2+\sigma_2^2)}} \text{exp} \left( - \frac{(\mu_1 - \mu_2)^2}{2(\sigma_1^2 + \sigma_2^2)} \right)
\end{equation}

Retonando a la integral
\begin{equation}
\begin{split}
I & = \int N(x;\mu_{\times},\sigma_{\times}^2) \overbrace{N(\mu_1;\mu_2,\sigma_1^2 + \sigma_2^2)}^{\text{Escalar independiente de x}} dx \\[0.3cm]
& = N(\mu_1;\mu_2,\sigma_1^2 + \sigma_2^2) \underbrace{\int N(x,\mu_{\times},\sigma_{\times}^2)  dx}_{\text{Integra 1}} \\
& = N(\mu_1;\mu_2,\sigma_1^2 + \sigma_2^2)
\end{split}
\end{equation}



