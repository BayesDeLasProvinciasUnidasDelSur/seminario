\documentclass[a4paper,10pt]{article}
\usepackage[utf8]{inputenc}
\usepackage[english]{babel}
\usepackage{tikz}
\usepackage{caption}
\usepackage{float} % para que los gr\'aficos se queden en su lugar con [H]
\usepackage{subcaption}

\usepackage[utf8]{inputenc}
\usepackage{url}
\usepackage{graphicx}
\usepackage{color}
\usepackage{amsmath} %para escribir funci\'on partida , matrices
\usepackage{amsthm} %para numerar definciones y teoremas
\usepackage[hidelinks]{hyperref} % para inlcuir links dentro del texto
%\usepackage{tabu} 
\usepackage{comment}
\usepackage{amsfonts} % \mathbb{N} -> conjunto de los n\'umeros naturales  
\usepackage{enumerate}
\usepackage{listings}
\usepackage[colorinlistoftodos, textsize=small]{todonotes} % Para poner notas en el medio del texto!! No olvidar hacer. 
\usepackage{framed} % Para encuadrar texto. \begin{framed}
\usepackage{csquotes} % Para citar texto \begin{displayquote}
\usepackage{varioref}
\usepackage{bm} % \bm{\alpha} bold greek symbol
\usepackage{pdfpages} % \includepdf
\usepackage[makeroom]{cancel} % \cancel{} \bcancel{} etc
\usepackage{wrapfig} % \begin{wrapfigure} Pone figura al lado del texto
\usepackage{mdframed}
\usepackage{algorithm}
\usepackage{quoting}
\usepackage{mathtools}	
\usepackage{paracol}

\newcommand{\vm}[1]{\mathbf{#1}}
\newcommand{\N}{\mathcal{N}}
\newcommand{\citel}[1]{\cite{#1}\label{#1}}
\newcommand\hfrac[2]{\genfrac{}{}{0pt}{}{#1}{#2}} %\frac{}{} sin la linea del medio

\newtheorem{midef}{Definition}
\newtheorem{miteo}{Theorem}
\newtheorem{mipropo}{Proposition}

\theoremstyle{definition}
\newtheorem{definition}{Definition}[section]
\newtheorem{theorem}{Theorem}[section]
\newtheorem{proposition}{Proposition}[section]


%http://latexcolor.com/
\definecolor{azul}{rgb}{0.36, 0.54, 0.66}
\definecolor{rojo}{rgb}{0.7, 0.2, 0.116}
\definecolor{rojopiso}{rgb}{0.8, 0.25, 0.17}
\definecolor{verdeingles}{rgb}{0.12, 0.5, 0.17}
\definecolor{ubuntu}{rgb}{0.44, 0.16, 0.39}
\definecolor{debian}{rgb}{0.84, 0.04, 0.33}
\definecolor{dkgreen}{rgb}{0,0.6,0}
\definecolor{gray}{rgb}{0.5,0.5,0.5}
\definecolor{mauve}{rgb}{0.58,0,0.82}

\lstset{
  language=Python,
  aboveskip=3mm,
  belowskip=3mm,
  showstringspaces=true,
  columns=flexible,
  basicstyle={\small\ttfamily},
  numbers=none,
  numberstyle=\tiny\color{gray},
  keywordstyle=\color{blue},
  commentstyle=\color{dkgreen},
  stringstyle=\color{mauve},
  breaklines=true,
  breakatwhitespace=true,
  tabsize=4
}

% tikzlibrary.code.tex
%
% Copyright 2010-2011 by Laura Dietz
% Copyright 2012 by Jaakko Luttinen
%
% This file may be distributed and/or modified
%
% 1. under the LaTeX Project Public License and/or
% 2. under the GNU General Public License.
%
% See the files LICENSE_LPPL and LICENSE_GPL for more details.

% Load other libraries

%\newcommand{\vast}{\bBigg@{2.5}}
% newcommand{\Vast}{\bBigg@{14.5}}
% \usepackage{helvet}
% \renewcommand{\familydefault}{\sfdefault}

\usetikzlibrary{shapes}
\usetikzlibrary{fit}
\usetikzlibrary{chains}
\usetikzlibrary{arrows}

% Latent node
\tikzstyle{latent} = [circle,fill=white,draw=black,inner sep=1pt,
minimum size=20pt, font=\fontsize{10}{10}\selectfont, node distance=1]
% Observed node
\tikzstyle{obs} = [latent,fill=gray!25]
% Invisible node
\tikzstyle{invisible} = [latent,minimum size=0pt,color=white, opacity=0, node distance=0]
% Constant node
\tikzstyle{const} = [rectangle, inner sep=0pt, node distance=0.1]
%state
\tikzstyle{estado} = [latent,minimum size=8pt,node distance=0.4]
%action
\tikzstyle{accion} =[latent,circle,minimum size=5pt,fill=black,node distance=0.4]
\tikzstyle{fijo} =[latent,circle,minimum size=5pt,fill=black]


% Factor node
\tikzstyle{factor} = [rectangle, fill=black,minimum size=10pt, draw=black, inner
sep=0pt, node distance=1]
% Deterministic node
\tikzstyle{det} = [latent, rectangle]

% Plate node
\tikzstyle{plate} = [draw, rectangle, rounded corners, fit=#1]
% Invisible wrapper node
\tikzstyle{wrap} = [inner sep=0pt, fit=#1]
% Gate
\tikzstyle{gate} = [draw, rectangle, dashed, fit=#1]

% Caption node
\tikzstyle{caption} = [font=\footnotesize, node distance=0] %
\tikzstyle{plate caption} = [caption, node distance=0, inner sep=0pt,
below left=5pt and 0pt of #1.south east] %
\tikzstyle{factor caption} = [caption] %
\tikzstyle{every label} += [caption] %

\tikzset{>={triangle 45}}

%\pgfdeclarelayer{b}
%\pgfdeclarelayer{f}
%\pgfsetlayers{b,main,f}

% \factoredge [options] {inputs} {factors} {outputs}
\newcommand{\factoredge}[4][]{ %
  % Connect all nodes #2 to all nodes #4 via all factors #3.
  \foreach \f in {#3} { %
    \foreach \x in {#2} { %
      \path (\x) edge[-,#1] (\f) ; %
      %\draw[-,#1] (\x) edge[-] (\f) ; %
    } ;
    \foreach \y in {#4} { %
      \path (\f) edge[->,#1] (\y) ; %
      %\draw[->,#1] (\f) -- (\y) ; %
    } ;
  } ;
}

% \edge [options] {inputs} {outputs}
\newcommand{\edge}[3][]{ %
  % Connect all nodes #2 to all nodes #3.
  \foreach \x in {#2} { %
    \foreach \y in {#3} { %
      \path (\x) edge [->,#1] (\y) ;%
      %\draw[->,#1] (\x) -- (\y) ;%
    } ;
  } ;
}

% \factor [options] {name} {caption} {inputs} {outputs}
\newcommand{\factor}[5][]{ %
  % Draw the factor node. Use alias to allow empty names.
  \node[factor, label={[name=#2-caption]#3}, name=#2, #1,
  alias=#2-alias] {} ; %
  % Connect all inputs to outputs via this factor
  \factoredge {#4} {#2-alias} {#5} ; %
}

% \plate [options] {name} {fitlist} {caption}
\newcommand{\plate}[4][]{ %
  \node[wrap=#3] (#2-wrap) {}; %
  \node[plate caption=#2-wrap] (#2-caption) {#4}; %
  \node[plate=(#2-wrap)(#2-caption), #1] (#2) {}; %
}

% \gate [options] {name} {fitlist} {inputs}
\newcommand{\gate}[4][]{ %
  \node[gate=#3, name=#2, #1, alias=#2-alias] {}; %
  \foreach \x in {#4} { %
    \draw [-*,thick] (\x) -- (#2-alias); %
  } ;%
}

% \vgate {name} {fitlist-left} {caption-left} {fitlist-right}
% {caption-right} {inputs}
\newcommand{\vgate}[6]{ %
  % Wrap the left and right parts
  \node[wrap=#2] (#1-left) {}; %
  \node[wrap=#4] (#1-right) {}; %
  % Draw the gate
  \node[gate=(#1-left)(#1-right)] (#1) {}; %
  % Add captions
  \node[caption, below left=of #1.north ] (#1-left-caption)
  {#3}; %
  \node[caption, below right=of #1.north ] (#1-right-caption)
  {#5}; %
  % Draw middle separation
  \draw [-, dashed] (#1.north) -- (#1.south); %
  % Draw inputs
  \foreach \x in {#6} { %
    \draw [-*,thick] (\x) -- (#1); %
  } ;%
}

% \hgate {name} {fitlist-top} {caption-top} {fitlist-bottom}
% {caption-bottom} {inputs}
\newcommand{\hgate}[6]{ %
  % Wrap the left and right parts
  \node[wrap=#2] (#1-top) {}; %
  \node[wrap=#4] (#1-bottom) {}; %
  % Draw the gate
  \node[gate=(#1-top)(#1-bottom)] (#1) {}; %
  % Add captions
  \node[caption, above right=of #1.west ] (#1-top-caption)
  {#3}; %
  \node[caption, below right=of #1.west ] (#1-bottom-caption)
  {#5}; %
  % Draw middle separation
  \draw [-, dashed] (#1.west) -- (#1.east); %
  % Draw inputs
  \foreach \x in {#6} { %
    \draw [-*,thick] (\x) -- (#1); %
  } ;%
}


\usepackage{amsmath} %para escribir funci\'on partida , matrices, hfrac
%
\newcommand{\gray}{\color{black!65}}

\usepackage{fullpage}
\newif\ifen
\newif\ifes
\newif\iffr
\newcommand{\en}[1]{\ifen#1 \fi}
\newcommand{\es}[1]{\ifes#1 \fi}
\newcommand{\fr}[1]{\iffr#1 \fi}
\newcommand{\En}[1]{\ifen#1\fi}
\newcommand{\Es}[1]{\ifes#1\fi}

\entrue

%opening

\title{\vspace{-1cm}
\en{Intersubjective agreements in contexts of uncertainty\\ \Large Life bets}\es{Acuerdos intersubjetivos en contextos de incertidumbre \\ \Large Apuestas de vida}}
\author{\en{Plurinational Bayes}\es{Bayes Plurinacional}}
%\date{\en{January 30, 2023}\es{30 de enero del 2023}}
\date{}
\begin{document}
%
\maketitle
%
% \begin{abstract}
%
% \end{abstract}
%
% \section{Introduction}
% %
% \en{Causal reasoning, based on the Bayesian approach, answered the question of how to reach intersubjective agreements in contexts of uncertainty, a prerequisite for empirical truths.}%
% \fr{Le raisonnement causal, basé sur l'approche bayésienne, a répondu la question de comment parvenir à des accords intersubjectifs dans des contextes d'incertitude, fondement des vérités empiriques.}%
% %
% \en{And to top it off, it has proven to be ideal for the social sciences, as it allows causal models to be expressed using intuitive graphical methods, providing a common language to leverage the knowledge of local, non-academic experts.}%
% \fr{Pour couronner le tout, il s'est avéré idéal pour les sciences sociales car il permet d'exprimer des modèles causaux à travers des méthodes graphiques intuitives, mettant au centre les connaissances d'experts locaux non académiques.}%
% %
% \en{It is more than just a tool.}%
% \fr{C'est plus qu'un outil.}%
%

% Parrafo

\en{Science is a human institution that seeks truths: intersubjective agreements with intercultural (or universal) validity.}%
\es{La ciencia es una institución humana que tiene pretensión de alcanzar verdades: acuerdos intersubjetivos con validez intercultural, universal.}%
%
\en{The formal sciences (mathematics, logic) reach these agreements by deriving theorems within closed axiomatic systems.}%
\es{Las ciencias formales (matemática, lógica) alcanzan estos acuerdos derivando teoremas dentro de sistemas axiomáticos cerrados.}%
%
\en{However, the empirical sciences (from physics to the social sciences) must validate their propositions within open systems that always contain some degree of uncertainty.}%
\es{Sin embargo, las ciencias empíricas (desde la física hasta las ciencias sociales) deben validar sus proposiciones en sistemas abiertos que por definición contienen siempre algún grado de incertidumbre.}%
%
\en{Is it then possible to reach intersubjective agreements (``truths'') in the empirical sciences if it is inevitable to say ``I don't know''?}%
\es{¿Es posible entonces alcanzar acuerdos intersubjetivos (``verdades'') en las ciencias empíricas si es inevitable decir ``no sé''?}%
%
\en{Yes.}%
\es{Sí.}%
%
\en{In short, we can avoid lying: not say more than what is known while incorporating all the available information.}%
\es{En pocas palabras, podemos evitar mentir: no decir más de lo que se sabe, incorporando al mismo tiempo toda la información disponible.}%

% Parrafo

\en{For example, suppose we know that there is one gift hidden behind a box among three.}%
\es{Por ejemplo, supongamos que sabemos que hay un regalo escondido detrás de una caja entre tres.}%
%
\en{Where is the gift?}%
\es{¿Dónde está el regalo?}%
%
\en{It is in one of the three boxes, but we don't know in which one.}%
\es{Está en alguna de las tres cajas, pero no sabemos en cuál de ellas.}%
%
\en{If we choose any of the boxes we would be stating more than we know, because we have no information that would make us prefer any of them.}%
\es{Si elegimos alguna de las cajas estaríamos afirmando más de lo que sabemos, porque no tenemos información que nos haga preferir ninguna de ellas.}%
%
\en{In this case, where we have no further information, we will agree on the need to divide belief into equal parts.}%
\es{En este caso, en los que no tenemos más información, vamos a estar de acuerdo en la necesidad de dividir la creencia en partes iguales.}%
%
\en{By maximizing uncertainty we avoid saying more than we know, allowing us to reach a first intersubjective agreement in contexts of uncertainty!}%
\es{Maximizando la incertidumbre evitamos decir más de lo que sabemos, permitiéndonos alcanzar un primer acuerdo intersubjetivo en contextos de incertidumbre!}%
%
\en{Great. But then, how do we preserve this intersubjective agreement if we receive new information?}%
\es{Bien!, ¿Pero después, cómo preservar el acuerdo intersubjetivo en casos más complejos, cuando recibimos nueva información?}%

% Parrafo
%
% \begin{figure}[ht!]
%     \centering
%     \begin{subfigure}[b]{0.48\textwidth}
%     \includegraphics[width=\linewidth]{../../aux/static/tsimane.jpg}
%     \caption{}
%     \label{fig:multilevel-selection-7}
%     \end{subfigure}
%     \caption{}
%     \label{fig:growth_rate_defector_mixed}
% \end{figure}


\en{The logic of uncertainty (probability theory) has been repeatedly derived from different principles (axioms), always arriving at the same two simple rules.}%
\es{La lógica con incertidumbre (teoría de la probabilidad) ha sido derivada repetidas veces a partir de diversos principios (axiomas), llegando siempre a las mismas dos simples reglas.}%
%
\en{The \emph{sum rule} ensures that we do not lose belief when we distribute it among mutually contradictory hypotheses: by adding up how much we assign to each hypothesis, we recover the initial 100\%.}%
\es{La regla de la suma garantiza no perder creencia cuando la distribuimos entre hipótesis mutuamente contradictorias: al sumar cuánto le hemos asignado a cada hipótesis, recuperamos el 100\% inicial.}%
%
% \en{Basic, isn't it?}%
% \es{Básico, no?}%
%
\en{And the (conditional or) \emph{product rule}, guarantees the coherence of beliefs with the available information: when we observe new data we preserve the prior belief that is still compatible with it (and the surviving belief is our new 100\%).}%
\es{Y la regla del producto (o condicional), garantiza la coherencia de las creencias con la información disponible: cuando observamos un nuevo dato preservamos la creencia previa que sigue siendo compatible (y la creencia que sobrevive es nuestro nuevo 100\%).}%
%
\en{Unlike ad-hoc approaches that select a single hypothesis (e.g.~by maximum likelihood), the strict application of probability rules (Bayesian approach), by believing at the same time in mutually contradictory hypotheses (A and not A), allows surprise, the only source of information, to be the sole filter of prior beliefs.}%
\es{A diferencia de los enfoques ad-hoc que seleccionan una única hipótesis (e.g.~por máxima verosimilitud), la aplicación estricta de las reglas de la probabilidad (enfoque bayesiano), al creer al mismo tiempo en hipótesis mutuamente contradictorias (A y no A), permite que sea la sorpresa, única fuente de información, el único filtro de las creencias previas.}%

% Parrafo

\en{The value of truth is not abstract, it is pragmatic.}%
\es{El valor de la verdad no es abstracto, es pragmático.}%
%
\en{The ``lie'' is a cross-cultural concept and its negation ``not to lie'' is a principle present in all societies of the world not because there is a striking coincidence between different particular moral criteria, but because truth has a concrete practical value, which is universal.}%
\es{La ``mentira'' es un concepto intercultural y su negación ``no mentir'' es un principio presente en todas las sociedades del mundo no porque haya una extraña coincidencia entre los distintos criterios morales particulares, sino porque la verdad tiene un valor práctico concreto, que es universal.}%
%
\en{The Bayesian selection process of alternative hypotheses (based on the sequence of surprises) is, like the evolutionary selection process (sequence of reproduction and survival rates), of a multiplicative nature.}%
\es{El proceso bayesiano de selección de las hipótesis alternativas (basado en la secuencia de sorpresas) es, como el proceso de selección evolutiva (secuencia de tasas de reproducción y supervivencia), de naturaleza multiplicativa.}%
%
\en{Because in them the impacts of losses are stronger than those of gains (for example, a single zero in the sequence produces an irreversible extinction) there is an advantage in favor of variants (hypotheses or life forms) that reduce fluctuations.}%
\es{Debido a que en ellos los impactos de las pérdidas son más fuertes que los de las ganancias (por ejemplo, un único cero en la secuencia produce una extinción irreversible) existe una ventaja a favor de las variantes (hipótesis o formas de vida) que reducen las fluctuaciones.}%
%
\en{This multiplicative structure of both selection processes has concrete consequences for life and knowledge.}%
\es{Esta estructura multiplicativa de los procesos de selección tiene consecuencias concretas para la vida y el conocimiento.}%

% Parrafo

\en{Since its origin, life has acquired an extraordinary complexity in terms of diversification, cooperation, specialization and coexistence.}%
\es{Desde su origen, la vida adquirió una extraordinaria complejidad en términos diversificación, cooperación, especialización y coexistencia.}%
%
\en{Why does that happen?}%
\es{¿Por qué ocurre eso?}%
%
\en{First, diversification makes it possible to individually reduce fluctuations in multiplicative processes.}%
\es{Primero, la diversificación permite reducir individualmente las fluctuaciones en los procesos multiplicativos.}%
%
\en{This property allows probability theory to acquire knowledge about the world, because evaluating individual hypotheses on the basis of the product of surprises produces an advantage in favor of hypotheses that diversify resources (probabilities) in the same proportion as the observed frequency.}%
\es{Esta propiedad le permite a la teoría de la probabilidad adquirir conocimiento sobre el mundo, pues evaluar las hipótesis individuales en base al producto de las sorpresas produce una ventaja a favor de las hipótesis que diversifican los recursos (probabilidades) en la misma proporción que la frecuencia observada.}%
%
\en{Moreover, through cooperation fluctuations are further reduced, allowing higher level hypotheses (causal models, paradigms) to ``emerge'' from the data because, by making predictions with the contribution of all its component hypotheses, it produces better results than any of them alone.}%
\es{Pero además, a través de la cooperación se reducen aún más las fluctuaciones, permitiendo que hipótesis de nivel superior (modelos causales, paradigmas) "emergan" de los datos porque, al realizar las predicciones con el aporte de todas las hipótesis que la componen, produce mejores resultados que cualquiera de ellas solas.}%
% \en{defectors is detrimental in the long run because it leads to an increase in the group's fluctuations and therefore also in their own.}%
% \es{los desertores se ve perjudicado a largo plazo debido a que produce un aumento en las fluctuaciones del grupo y por lo tanto también en las suyas.}%
% %
\en{And finally, since with cooperation it is no longer necessary to reduce fluctuations by individual diversification, an advantage in favor of specialization arises producing the emergence of worldviews, such as science and culture, where highly specialized domain-specific paradigms coexist.}%
\es{Y finalmente, debido a que con la cooperación deja de ser necesario reducir fluctuaciones por diversificación individual, surge una ventaja a favor de la especialización produciendo la emergencia de cosmovisiones, como la ciencia y la cultura, donde coexisten paradigmas altamente especializados a dominios particulares.}%
% el conocimiento empírico también depende de que varios niveles de cooperación: las hipótesis cooperando para formar variables, las variables cooperendo para formar modelos causales, y conjuntos de modelos cooperendo para formar teorías (o culturas).

% Parrafo [Con la vida pasa algo parecido]

\en{In the same way, the multiplicative selection of life (by sequences of reproduction and survival rates) has produced throughout its history a series of evolutionary transitions in which entities capable of self-replication after the transition became part of an indissoluble cooperative unit.}%
\es{De la misma forma, la selección multiplicativa de la vida (por secuencias de tasas de reproducción y supervivencia) ha producido a lo largo de su historia una serie de transiciones evolutivas en las que entidades capaces de autorreplicación luego de la transición pasaron a formar parte de unidades cooperativas indisolubles.}%
%
\en{Our own life depends on several levels of cooperation (with specialization), without which we are not able to survive: the union of our cells with mitochondria and the emergence of organelles; our multicellular organism and the emergence of organs; our society and the emergence of roles and groups; the coexistence between species and the emergence of ecosystems.}%
\es{Nuestra propia vida depende de varios niveles de cooperación (con especialización), sin los cuales no somos capaces de sobrevivir: la unión de nuestras células con las mitocondrias y la emergencia de las organelas; nuestro organismo multicelular y la emergencia de los órganos; nuestra sociedad y la emergencia de los roles y grupos; la coexistencia entre especies y la emergencia de los ecosistemas.}%
%
\en{For our species, the emergence of mutual understanding produced a radical change.}%
\es{Para nuestra especie, el surgimiento de la comprensión mutua produjo un cambio radical.}%
%
\en{Before the cultural transition, we were in serious danger of extinction, as evidenced by the low diversity of the human genome.}%
\es{Antes de la transición cultural, estábamos en grave peligro de extinción, lo que se evidencia en la baja diversidad del genoma humano.}%
%
\en{But when knowledge, that previously had to be rediscovered individually, became a common resource passed on from generation to generation, we were able to occupy all the ecological niches of the earth as no other terrestrial vertebrate had done before.}%
\es{Pero cuando el conocimiento, que antes debía ser redescubierto individualmente, pasó a ser un recurso común transmitido de generación en generación, fuimos capaces de ocupar todos los nichos ecológicos de la tierra como ningún otro vertebrado terrestre lo había logrado antes.}

% Parrafo

\en{The emergence of higher level units is a permanent phenomenon in the history of life and knowledge because, under multiplicative selection processes, the defecting variants negatively affect their own long-term growth rate: this behavior, by increasing the fluctuations on whom it depends (the cooperators), also increases their own fluctuations.}%
\es{La emergencia de unidades de nivel superior es un fenómeno permanente en la historia de la vida y del conocimiento debido a que, bajo procesos de selección multiplicativa las variantes desertoras afectan negativamente su propia tasa de crecimiento a largo plazo: este comportamiento, al aumentar las fluctuaciones que en quienes depende (las cooperadoras), aumenta también sus propias fluctuaciones.}%
%
\en{The experience accumulated by the most diverse communities in the world has led, independently, to a universal obligation to give and receive, and to the development of technologies of reciprocity that reactivate community bonds through (festive or coercive) exchange rites.}%
\es{La experiencia acumulada por las comunidades más diversas del mundo ha llevado, de manera independiente, a una obligación universal de dar y recibir, y al desarrollo de tecnologías de reciprocidad que reactivan los vínculos comunitarios a través de ritos de intercambio (festivos o coercitivos).}%
%
\en{In turn, the technologies of ecological reciprocity produced the independent emergence of domestication of animals and plants, the development of special genetic traits that arise as a result of prolonged symbiotic interaction between species.}
\es{A su vez, las tecnologías de reciprocidad ecológica produjeron la aparición independiente de la domesticación de animales y vegetales, el desarrollo de caracteres genéticos especiales que surgen como resultado de una interacción simbiótica prolongada entre las especies.}%

% Parrafo

\en{Agriculture developed in parallel in the six geographic systems of the earth (sub-Saharan Africa, Wester Asia, China, Oceania, North America and South America), creating the main population and technological centers of humanity.}%
\es{La agricultura se desarrolló de forma paralela en los seis grandes sistemas geográficos de la tierra (África subsahariana, Asia del Este, China, Oceanía, América del Norte y América del Sur), creando los principales centros poblacionales y tecnológicos de la humanidad.}%
%
\en{During the 1400s, prosperous societies flourished around the world.}%
\es{Durante el año 1400 el mundo florecía de sociedades prósperas.}%
%
\en{The Pacific Ocean was already fully occupied, and there had already been exchanges between Oceania and South America, which is evident in the genetics of current populations.}%
\es{El océano Pacífico ya estaba totalmente ocupado, y ya se habían producido intercambios entre Oceanía y América del Sur, que se evidencia en la genética de las poblaciones actuales.}%
%
\en{China was still the main productive and technological center of the world after two thousand years, and in the Arab world these products were traded from the Pacific Ocean in the Philippines to the Atlantic Ocean in Spain.}%
\es{China seguía siendo el principal centro productivo y tecnológico del mundo después de 2 mileños, y en el mundo Árabe se comerciaban esos productos desde el océano Pacífico en las Filipinas hasta el océano Atlántico en España.}%
%
\en{And in the center of human genetic and cultural diversity, sub-Saharan Africa, the Bantu society developed among others.}%
\es{Y en el centro de la diversidad genética y cultural humana, África subsahariana, la sociedad Bantu se desarrollaba entre otras.}%

% Parrafo

\en{The breakdown of the cooperative pact negatively affects those who promote it without the need to introduce punishments.}%
\es{La ruptura del pacto cooperativo afecta negativamente a quienes lo promueven sin necesidad de introducir sanciones.}%
%
\en{After the massive destruction of the cultural diversity produced by the Roman Empire, Western Europe entered, against the rest of the world, into a long process of cultural involution and internal violence known as the ``Middle Ages''.}%
\es{Luego de la masiva destrucción de la diversidad cultural que produce el imperio Romano en su entorno, Europa occidental entre, a contramano del resto del mundo, en un largo proceso de involución cultural y de violencia interna conocido como ``Edad media''.}%
%
\en{Like any society that lives in a terrible socioeconomic situation, as soon as feudal Europe discovers how to navigate the Atlantic, a massive migratory process begins.}%
\es{Como toda sociedad que vive una pésima situación socioeconómica, apenas la Europa feudal descubre cómo navegar el Atlántico, comienza un masivo proceso migratorio.}%
%
\en{The coincidence of a series of events placed this historically marginal society in a situation of worldwide privilege.}%
\es{La coincidencia de un conjunto de eventos puso a esta sociedad, históricamente marginal, en una situación de privilegio mundial.}%
%
\en{A century earlier, China had incorporated silver as one of its official currencies, and the massive arrival of feudal immigrants to South America produced a series of epidemics that decimated its population.}%
\es{Un siglo antes, China había incorporado la plata como una de sus monedas oficiales, y la llegada masiva de inmigrantes feudales a América del Sur produjo una serie de epidemias que diezmó su población.}%
%
\en{But the geopolitical shift did not occur until feudal explorers discovered the silver mountain of Potosi in 1546: 25 years later feudal Europe dominated the Mediterranean (battle of Lepanto) and the cycle of foreign technology imports began, mainly from China.}%
\es{Pero el cambio geopolítico no se produjo hasta que los exploradores feudales descubrieron la montaña de plata de Potosí en 1546: 25 años después, la Europa feudal dominó el Mediterráneo (batalla de Lepanto) y comenzó el ciclo de importaciones de tecnología extranjera, principalmente de China.}%

% Parrafo

\en{Even during the 1800s, after the American independences, Western Europe still had a trade deficit with China, which since the mid-1700s had been financed in part by the opium drug trade.}%
\es{Todavía durante los años 1800, luego de las independencias de Ámerica, Europa occidental seguía teniendo déficit comercial con China, el que desde mediados de 1700 financiaba en parte mediante el narcotráfico de opio.}%
%
\en{The consequences were more severe when a cheaper and more potent opium mixture was developed in 1818.}%
\es{Las consecuencias se agravaron cuando en 1818 se desarrolló una mezcla de opio más barata y potente.}%
%
\en{The number of addicts became alarming, and in 1839 China made the mistake of declaring war against the British narco-state in its own territory.}%
\es{El número de adictos llegó a ser lo suficientemente alarmante, y en 1839 China comete el error de declarar la guerra al narco-estado británico en su propio territorio.}%
%
\en{The results were terrible: China lost 1/5 of its population and was subjected to foreign invasions for a century.}%
\es{Los resultados fueron terribles: China pierde 1/5 de su población y queda sumida durante un siglo de invasiones extranjeras.}%
%
\en{It was only after the defeat of China that Western Europe began in 1850 to occupy Africa and the still autonomous territories of America.}%
\es{Solo después de la derrota de China es que Europa occidenta comienza en 1850 la ocupación de África y de los territorios de América todavía autónomos.}%
%
\en{The era of genocide and massive loss of cultural diversity begins.}%
\es{Comienza la era de genocidios y masiva pérdida de diversidad cultural.}%

% Parrafo

\en{Just as the decadence of feudal society was preceded by a massive loss of cultural diversity in the region of the Roman Empire, the current ecological crisis has been preceded by a massive loss of global cultural diversity.}%
\es{Así como la decadencia de la sociedad feudal estuvo precedida por una masiva pérdida de diversidad cultural en la región del imperio Romano, la crísis ecológica actual ha estado precedida por una masiva pérdida de la diversidad cultural global.}%
%
\en{Despite all the advances, metropolitan science was not able to compensate for the loss of millenarian knowledge caused by colonial-modernity and as a consequence the ecological crisis continues to deepen.}%
\es{A pesar de todos los avances, la ciencia metropolitana no fue capaz de compensar la pérdida de conocimientos milenarios provocada por la modernidad-colonial y como consecuencia la crisis ecológica no deja de profundizarse.}%
%
\en{The advantage in favor of diversification and cooperation is not only theoretical, its breakdown has consequences for life and knowledge.}%
\es{La ventaja a favor de la diversificación y la cooperación no es solo teórica, su ruptura tiene consecuencia para la vida y el conocimiento.}%
%
\en{Just as the problem of overfitting is a direct consequence of selecting a single hypothesis, the current ecological crisis is a direct consequence of the imposition of a single type of society.}%
\es{Del mismo modo que el problema del sobreajuste es consecuencia directa de seleccionar una única hipótesis, la crisis ecológica actual es consecuencia directa de la imposición de un único tipo de sociedad.}%
%
\en{In the long term, only variants capable of reducing fluctuations through diversification and cooperation survive.}%
\es{A largo plazo solo sobreviven las variantes capaces de reducir las fluctuaciones por diversificación y cooperación.}%
% %
% \en{It seems that recovering coexistence with the ecological systems on which we depend requires recovering reciprocal coexistence between autonomous communities.}%
% \es{Parece que para recuperar la coexistencia con los sistemas ecológicos del que dependemos necesitamos recuperar la convivencia recíproca entre comunidades autónomas.}%

% Parrafo

\en{Just as the Bayesian approach adapts knowledge by believing at the same time in A and non-A, a Plurinational society adapts to life through reciprocal coexistence between autonomous communities.}%
\es{Así como el enfoque bayesiano adapta el conocimiento creyendo al mismo tiempo en A y no A, una sociedad plurinacional se adapta a la vida a través de la convivencia recíproca entre comunidades autónomas.}%

\end{document}
