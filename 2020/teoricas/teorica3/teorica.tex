\documentclass[shownotes,aspectratio=169]{beamer}
\input{../../../aux/tex/diapo_encabezado.tex}
% tikzlibrary.code.tex
%
% Copyright 2010-2011 by Laura Dietz
% Copyright 2012 by Jaakko Luttinen
%
% This file may be distributed and/or modified
%
% 1. under the LaTeX Project Public License and/or
% 2. under the GNU General Public License.
%
% See the files LICENSE_LPPL and LICENSE_GPL for more details.

% Load other libraries

%\newcommand{\vast}{\bBigg@{2.5}}
% newcommand{\Vast}{\bBigg@{14.5}}
% \usepackage{helvet}
% \renewcommand{\familydefault}{\sfdefault}

\usetikzlibrary{shapes}
\usetikzlibrary{fit}
\usetikzlibrary{chains}
\usetikzlibrary{arrows}

% Latent node
\tikzstyle{latent} = [circle,fill=white,draw=black,inner sep=1pt,
minimum size=20pt, font=\fontsize{10}{10}\selectfont, node distance=1]
% Observed node
\tikzstyle{obs} = [latent,fill=gray!25]
% Invisible node
\tikzstyle{invisible} = [latent,minimum size=0pt,color=white, opacity=0, node distance=0]
% Constant node
\tikzstyle{const} = [rectangle, inner sep=0pt, node distance=0.1]
%state
\tikzstyle{estado} = [latent,minimum size=8pt,node distance=0.4]
%action
\tikzstyle{accion} =[latent,circle,minimum size=5pt,fill=black,node distance=0.4]
\tikzstyle{fijo} =[latent,circle,minimum size=5pt,fill=black]


% Factor node
\tikzstyle{factor} = [rectangle, fill=black,minimum size=10pt, draw=black, inner
sep=0pt, node distance=1]
% Deterministic node
\tikzstyle{det} = [latent, rectangle]

% Plate node
\tikzstyle{plate} = [draw, rectangle, rounded corners, fit=#1]
% Invisible wrapper node
\tikzstyle{wrap} = [inner sep=0pt, fit=#1]
% Gate
\tikzstyle{gate} = [draw, rectangle, dashed, fit=#1]

% Caption node
\tikzstyle{caption} = [font=\footnotesize, node distance=0] %
\tikzstyle{plate caption} = [caption, node distance=0, inner sep=0pt,
below left=5pt and 0pt of #1.south east] %
\tikzstyle{factor caption} = [caption] %
\tikzstyle{every label} += [caption] %

\tikzset{>={triangle 45}}

%\pgfdeclarelayer{b}
%\pgfdeclarelayer{f}
%\pgfsetlayers{b,main,f}

% \factoredge [options] {inputs} {factors} {outputs}
\newcommand{\factoredge}[4][]{ %
  % Connect all nodes #2 to all nodes #4 via all factors #3.
  \foreach \f in {#3} { %
    \foreach \x in {#2} { %
      \path (\x) edge[-,#1] (\f) ; %
      %\draw[-,#1] (\x) edge[-] (\f) ; %
    } ;
    \foreach \y in {#4} { %
      \path (\f) edge[->,#1] (\y) ; %
      %\draw[->,#1] (\f) -- (\y) ; %
    } ;
  } ;
}

% \edge [options] {inputs} {outputs}
\newcommand{\edge}[3][]{ %
  % Connect all nodes #2 to all nodes #3.
  \foreach \x in {#2} { %
    \foreach \y in {#3} { %
      \path (\x) edge [->,#1] (\y) ;%
      %\draw[->,#1] (\x) -- (\y) ;%
    } ;
  } ;
}

% \factor [options] {name} {caption} {inputs} {outputs}
\newcommand{\factor}[5][]{ %
  % Draw the factor node. Use alias to allow empty names.
  \node[factor, label={[name=#2-caption]#3}, name=#2, #1,
  alias=#2-alias] {} ; %
  % Connect all inputs to outputs via this factor
  \factoredge {#4} {#2-alias} {#5} ; %
}

% \plate [options] {name} {fitlist} {caption}
\newcommand{\plate}[4][]{ %
  \node[wrap=#3] (#2-wrap) {}; %
  \node[plate caption=#2-wrap] (#2-caption) {#4}; %
  \node[plate=(#2-wrap)(#2-caption), #1] (#2) {}; %
}

% \gate [options] {name} {fitlist} {inputs}
\newcommand{\gate}[4][]{ %
  \node[gate=#3, name=#2, #1, alias=#2-alias] {}; %
  \foreach \x in {#4} { %
    \draw [-*,thick] (\x) -- (#2-alias); %
  } ;%
}

% \vgate {name} {fitlist-left} {caption-left} {fitlist-right}
% {caption-right} {inputs}
\newcommand{\vgate}[6]{ %
  % Wrap the left and right parts
  \node[wrap=#2] (#1-left) {}; %
  \node[wrap=#4] (#1-right) {}; %
  % Draw the gate
  \node[gate=(#1-left)(#1-right)] (#1) {}; %
  % Add captions
  \node[caption, below left=of #1.north ] (#1-left-caption)
  {#3}; %
  \node[caption, below right=of #1.north ] (#1-right-caption)
  {#5}; %
  % Draw middle separation
  \draw [-, dashed] (#1.north) -- (#1.south); %
  % Draw inputs
  \foreach \x in {#6} { %
    \draw [-*,thick] (\x) -- (#1); %
  } ;%
}

% \hgate {name} {fitlist-top} {caption-top} {fitlist-bottom}
% {caption-bottom} {inputs}
\newcommand{\hgate}[6]{ %
  % Wrap the left and right parts
  \node[wrap=#2] (#1-top) {}; %
  \node[wrap=#4] (#1-bottom) {}; %
  % Draw the gate
  \node[gate=(#1-top)(#1-bottom)] (#1) {}; %
  % Add captions
  \node[caption, above right=of #1.west ] (#1-top-caption)
  {#3}; %
  \node[caption, below right=of #1.west ] (#1-bottom-caption)
  {#5}; %
  % Draw middle separation
  \draw [-, dashed] (#1.west) -- (#1.east); %
  % Draw inputs
  \foreach \x in {#6} { %
    \draw [-*,thick] (\x) -- (#1); %
  } ;%
}


 \mode<presentation>
 {
 %   \usetheme{Madrid}      % or try Darmstadt, Madrid, Warsaw, ...
 %   \usecolortheme{default} % or try albatross, beaver, crane, ...
 %   \usefonttheme{serif}  % or try serif, structurebold, ...
  \usetheme{Antibes}
  \setbeamertemplate{navigation symbols}{}
 }
 
\usepackage{todonotes}
\setbeameroption{show notes}

%\title[Bayes del Sur]{}

\begin{document}

 \color{black!85}

\small
\setbeamercolor{background canvas}{bg=gray!18}

\begin{frame}[plain,noframenumbering]
 
 \begin{textblock}{80}(0,0)
\begin{center}
 \huge  \textcolor{black!66}{Logia Bayesiana de las \\ Provincias Unidas del Sud}
\end{center}
\end{textblock}

 %\vspace{2cm}brown
%\maketitle
\Wider[2cm]{
\includegraphics[width=1\textwidth]{../../../aux/static/peligro_predador}
}
\end{frame}

\begin{frame}[plain]
\vspace{0.5cm}
\begin{center} 
 \LARGE Modelos con m\'ultiples param\'etros \\
 
 \large Cap\'itulo 3
\end{center}
\end{frame}

\begin{frame}[plain]
 \begin{textblock}{160}(0,0)
\begin{center}
 \Large Posteriori marginal
\end{center}
\end{textblock}
 \vspace{0.75cm}
 
 \only<1->{
 \begin{textblock}{160}(0,25)
 \begin{equation*}
  p(\theta_1,\theta_2|y) = \frac{p(y|\theta_1,\theta_2)p(\theta_1,\theta_2) }{p(y)}
 \end{equation*}
 \end{textblock}
 }
 
 \only<2->{
 \begin{textblock}{160}(0,45)
 \begin{equation*}
  p(\theta_1|y) = \int p(\theta_1,\theta_2|y) d\theta_2
 \end{equation*} 
 \end{textblock}
 }
 
 \begin{textblock}{160}(0,63)
 \begin{center}
 \only<3>{\textbf{Ejemplo}:  \ \ $p(y|\mu,\sigma) = \N(y| \mu,\sigma^2)$ \\}
 \end{center}
 \end{textblock}
 
\end{frame}

\begin{frame}[plain]
 \begin{textblock}{160}(0,0)
\begin{center}
 \Large Posteriori marginal \\
 \large Ejemplo
\end{center}
\end{textblock}
 \vspace{0.75cm}
 
 \only<1->{
 \begin{textblock}{160}(0,25)
 \begin{equation*}
  p(y|\mu,\sigma) = \N(y| \mu,\sigma^2)  \hspace{1cm} p(\mu, \sigma) \propto \sigma^{-2}
 \end{equation*}
 \end{textblock}
 }
% 
 \begin{textblock}{160}(0,35)
 \begin{align*}
\onslide<2->{ p(\mu,\sigma|\vm{y}) & \propto \sigma^{-n-2} \exp \left( -\frac{1}{2\sigma^2} \sum_{i=1}^{n} (\vm{y}_i - \mu)^2 \right) \\}
\onslide<3->{& = \sigma^{-n-2} \exp \left( -\frac{1}{2\sigma^2}  \left[ n(\overline{y}-\mu)^2 + \sum_{i=1}^n (\vm{y}_i - \overline{y}) \right] \right) \\}
\onslide<4->{& = \sigma^{-n-2} \exp \left( -\frac{1}{2\sigma^2}  \left[ n(\overline{y}-\mu)^2 + (n-1) s^2 \right] \right) \\
& \text{ con } s^2 = \frac{1}{n-1} \sum_{i=1}^n (\vm{y}_i - \overline{y})^2
}
 \end{align*}
 \end{textblock}

 
\end{frame}



\begin{frame}[plain]
 \begin{textblock}{160}(0,0)
\begin{center}
 \Large Posteriori marginal \\
 \large Ejemplo
\end{center}
\end{textblock}
 \vspace{0.75cm}
 
 \only<1->{
 \begin{textblock}{160}(0,25)
 \begin{equation*}
  p(y|\mu,\sigma) = \N(y| \mu,\sigma^2)  \hspace{1cm} p(\mu, \sigma) \propto \sigma^{-2}
 \end{equation*}
 \end{textblock}
 }

 \only<2->{
 \begin{textblock}{160}(0,35)
 \begin{equation*}
  p(\mu|\sigma,y) = \N(\mu|\overline{y},\sigma^2/n)
 \end{equation*}
 \end{textblock}
 }
 
  \only<3->{
 \begin{textblock}{160}(0,45)
 \begin{equation*}
  p(\sigma|y) = \text{Inv-}\chi^2(n-1,s^2)
 \end{equation*}
 \end{textblock}
 }

 \only<4->{
 \begin{textblock}{160}(0,55)
 \begin{equation*}
  p(\mu|y) = t_{n-1}(\overline{y},s^2/n)
 \end{equation*}
 \end{textblock}
 }

 \only<5->{
\begin{textblock}{160}(0,68)
\centering
Ocurre algo similar cuando los priors son 
\begin{align*}
 \mu|\sigma^2 \sim \N(\mu_0, \sigma^2/\kappa_0) \ \ \ \ \ \ \sigma \sim \text{Inv-}\chi^2(\nu_0,\sigma_0^2)
\end{align*}
 \end{textblock}
 }
\end{frame}


\begin{frame}[plain]
 \begin{textblock}{160}(0,0)
\begin{center}
 \Large Multinomial y Dirichlet \\
\end{center}
\end{textblock}
 \vspace{0.75cm}

 
 \pause
 Multinomial
 \begin{align*}
  p(\vm{y}|\vm{\theta}) \propto \prod_{j=1}^k \vm{\theta}_j^{\vm{y}_j}
 \end{align*}

 
 \pause
 Dirichlet
 \begin{align*}
  p(\vm{\theta}|\bm{\alpha}) \propto \prod_{j=1}^k \vm{\theta}_j^{\bm{\alpha}_j}
 \end{align*}

 
 
 
 \end{frame}

 

%  
%  \begin{textblock}{160}(0,35)
%  \begin{equation*}
%  \begin{split}
%  \only<2->{ p(\mu,\sigma|\bm{y}) &\propto p(\mu, \sigma) \prod_i^n p(\bm{y}_i|\mu,\sigma) \\}
%  \only<3->{&= } \sigma^{-n}\sigma^{-2} \exp\Big( \frac{1}{2\sigma^2} \sum_i^n (\bm{y}_i - \mu)^2 \Big) \\
%   \only<4->{&= } \sigma^{-(n+2)} \exp\Big( \frac{1}{2\sigma^2} [ n(\overline{y}-\mu)^2 + (n-1)s^2]\Big)
%  \end{split}
%  \end{equation*}
%  
%  con 
%  \end{textblock}
%  
 

 \begin{frame}[plain]
 \begin{textblock}{160}(0,0)
\begin{center}
 \Large Gaussiana multivariada
\end{center}
\end{textblock}


\begin{align*}
\N(\vm{x}|\bm{\mu},\vm{\Sigma}) = \frac{1}{(2\pi)^{D/2}} \frac{1}{|\vm{\Sigma}|^{1/2}} \exp\left( - \frac{1}{2} (\vm{x}-\bm{\mu})^T \vm{\Sigma}^{-1} (\vm{x}-\bm{\mu}) \right)
\end{align*}

\vspace{0.5cm}

\begin{itemize}
 \item[$\circ$] Con $\vm{x}$ y $\bm{\mu}$ vectores $D$-dimensionales
 \item[$\circ$] Con $\vm{\Sigma}$ matriz de covarianza $D \times D$
 \item[$\circ$] $|\Sigma|$ el determinante de $\Sigma$
\end{itemize}

 \end{frame}
 
\begin{frame}[plain]
 \begin{textblock}{160}(0,0)
\begin{center}
 \Large Distribuci\'on Gaussiana condicional y marginal.
\end{center}
\end{textblock}
\vspace{0.5cm}

Dado
\begin{align*}
 p(\vm{x}) = \N(\vm{x}|\bm{\mu},\vm{\Sigma})
\end{align*}

tal que
\begin{align*}
 \vm{x} = \begin{pmatrix} \vm{x}_a \\ \vm{x}_b \end{pmatrix} 
 \ \ \
 \bm{\mu} = \begin{pmatrix} \bm{\mu}_a \\ \bm{\mu}_b \end{pmatrix} 
\ \ \
 \vm{\Sigma} = \begin{pmatrix} \vm{\Sigma}_{aa} & \vm{\Sigma}_{ab} \\ \vm{\Sigma}_{ba} & \vm{\Sigma}_{bb} \end{pmatrix} 
 \ \ \ 
 \vm{\Sigma}^{-1} = \vm{\Lambda} = \begin{pmatrix} \vm{\Lambda}_{aa} & \vm{\Lambda}_{ab} \\ \vm{\Lambda}_{ba} & \vm{ \Lambda}_{bb} \end{pmatrix} 
 \end{align*}

Queremos encontrar,
\begin{align*}
 p(\vm{x}_a|\vm{x}_b) & = \N(\vm{x}_a|\bm{\mu}_{a|b},\vm{\Sigma}_{a|b}) \\
 p(\vm{x}_a) & = \N(\vm{x}_a|\bm{\mu}_a,\vm{\Sigma}_a)
\end{align*}

\only<2>{
 \begin{textblock}{160}(0,84)\centering \scriptsize
Notar que $\bm{\Lambda}_{aa} \neq \bm{\Sigma}_{aa}^{-1}$
 \end{textblock}
 }
\end{frame}

\begin{frame}[plain]
 \begin{textblock}{160}(0,0)
\begin{center}
 \Large Distribuci\'on Gaussiana condicional.
\end{center}
\end{textblock}
\vspace{0.75cm}

\onslide<1->{
\begin{textblock}{160}(0,12)
\begin{align*}
 p(\vm{x}_a|\vm{x}_b) \propto p(\vm{x}_a,\vm{x}_b) \propto \exp\Big( -\frac{1}{2} (\vm{x}-\bm{\mu})^T \vm{\Sigma}^{-1} (\vm{x}-\bm{\mu}) \Big)
\end{align*}
\end{textblock}
}

\only<2->{
\begin{textblock}{160}(0,28)
\begin{align*}
-\frac{1}{2}(\vm{x}-\bm{\mu})^T \vm{\Sigma}^{-1} (\vm{x}-\bm{\mu}) = -\frac{1}{2} \underbrace{\vm{x}^T \vm{\Sigma}^{-1} \vm{x}}_{\text{Cuadr\'atico}} +  \underbrace{\vm{x}^T \vm{\Sigma}^{-1} \bm{\mu}}_{\text{Lineal}} + \text{const} 
\end{align*}
\end{textblock}
}
 

\begin{textblock}{160}(0,46)
\begin{align*}
\onslide<3->{& -\frac{1}{2}(\vm{x}_a-\bm{\mu}_a)^T \vm{\Lambda}_{aa}^{-1} (\vm{x}_a-\bm{\mu}_a)
 -\frac{1}{2}(\vm{x}_a-\bm{\mu}_a)^T \vm{\Lambda}_{ab}^{-1} (\vm{x}_b-\bm{\mu}_b) \\
&  -\frac{1}{2}(\vm{x}_b-\bm{\mu}_b)^T \vm{\Lambda}_{ba}^{-1} (\vm{x}_a-\bm{\mu}_a)
-\frac{1}{2}(\vm{x}_b-\bm{\mu}_b)^T \vm{\Lambda}_{bb}^{-1} (\vm{x}_b-\bm{\mu}_b) \\
& = } \\
\only<3>{ & -\frac{1}{2} \vm{x}_a^T \, \vm{\Lambda}_{aa} \, \vm{x}_a + \vm{x}_a^T \big( \, \vm{\Lambda}_{aa} \bm{\mu}_a  - \vm{\Lambda}_{ab}(\vm{x}_b-\bm{\mu}_b) \, \big) + \text{const} \\} 
\onslide<4->{ & -\frac{1}{2} \vm{x}_a^T \underbrace{\vm{\Lambda}_{aa}}_{\vm{\Sigma}_{a|b}^{-1}} \vm{x}_a + \vm{x}_a^T \big( \underbrace{\vm{\Lambda}_{aa} \bm{\mu}_a  - \vm{\Lambda}_{ab}(\vm{x}_b-\bm{\mu}_b)}_{\vm{\Sigma}_{a|b}^{-1}\bm{\mu}_{a|b}} \big) + \text{const}}
\end{align*}
\end{textblock}



\end{frame}

\begin{frame}[plain]
 \begin{textblock}{160}(0,0)
\begin{center}
 \Large Distribuci\'on Gaussiana condicional
\end{center}
\end{textblock}
\vspace{0.75cm}

\begin{align*}
\vm{\Sigma}_{a|b} = \vm{\Lambda}_{aa}^{-1}
 \ \ \ \ \ \ 
 \bm{\mu}_{a|b} & = \vm{\Sigma}_{a|b} \Big( \vm{\Lambda}_{aa} \bm{\mu}_a  - \vm{\Lambda}_{ab}(\vm{x}_b-\bm{\mu}_b)  \Big) \\
 & = \bm{\mu}_a - \vm{\Lambda}_{aa}^{-1} \vm{\Lambda}_{ab}(\vm{x}_b-\bm{\mu}_b)
\end{align*}

\pause
\vspace{1cm}
\Wider[-7cm]{
\begin{mdframed}[backgroundcolor=black!20]
\[p(\vm{x}_a|\vm{x}_b) = \N(\vm{x}| \bm{\mu}_{a|b} , \vm{\Lambda}_{aa}^{-1} ) \]
\end{mdframed}
}


\end{frame}


\begin{frame}[plain]
 \begin{textblock}{160}(0,0)
\begin{center}
 \large Matriz de precisi\'on
\end{center}
\end{textblock}
\vspace{0.75cm}

\begin{align*}
 \begin{pmatrix} \vm{A} & \vm{B} \\ \vm{C} & \vm{D} \end{pmatrix}  
 = 
 \begin{pmatrix} 
 \vm{M} &  -\vm{M}\vm{B}\vm{D}^{-1}  \\
 -\vm{D}^{-1}\vm{C}\vm{M}  & \vm{D}^{-1} \vm{C}\vm{M}\vm{B}\vm{D}^{-1} 
 \end{pmatrix}
 \hspace{0.8cm} \text{con } \vm{M} = (\vm{A}-\vm{B}\vm{D}^{-1}\vm{C})^{-1} 
\end{align*}


\begin{align*}
 \vm{\Lambda}_{aa} & = ( \vm{\Sigma}_{aa} - \vm{\Sigma}_{ab} \vm{\Sigma}_{bb}^{-1} \vm{\Sigma}_{ba} )^{-1} \\
 \vm{\Lambda}_{ab} & = -( \vm{\Sigma}_{aa} - \vm{\Sigma}_{ab} \vm{\Sigma}_{bb}^{-1} \vm{\Sigma}_{ba} )^{-1}  \vm{\Sigma}_{ab}  \vm{\Sigma}_{bb}^{-1}
\end{align*}

% 
% \Wider[-6cm]{
% \begin{mdframed}[backgroundcolor=black!20]
% \vspace{-0.4cm}
% \begin{equation*}
% \begin{split}
%  \bm{\mu}_{a|b} & = \bm{\mu}_a + \vm{\Sigma}_{ab}  \vm{\Sigma}_{bb}^{-1} (\vm{x}_b-\bm{\mu}_b)\\
%  \bm{\Sigma}_{a|b} & = \vm{\Sigma}_{aa} - \vm{\Sigma}_{ab} \vm{\Sigma}_{bb}^{-1} \vm{\Sigma}_{ba} \\
%  \end{split}
% \end{equation*}
% \end{mdframed}
% }

\end{frame}

\begin{frame}[plain]
 \begin{textblock}{160}(0,0)
\begin{center}
 \large Demostraci\'on de la matriz inversa
\end{center}
\end{textblock}
\vspace{0.75cm}

\only<1->{
\begin{textblock}{160}(0,18)
\begin{align*}
 \begin{pmatrix} \vm{A} & \vm{B} \\ \vm{C} & \vm{D} \end{pmatrix}  
 \begin{pmatrix} \vm{A} & \vm{B} \\ \vm{C} & \vm{D} \end{pmatrix}^{-1}   = \begin{pmatrix} \vm{I} & \vm{O} \\ \vm{O} & \vm{I} \end{pmatrix}  
\end{align*}
\end{textblock}
}

\only<2->{
\begin{textblock}{160}(0,38)
\begin{align*}
 \begin{pmatrix} \vm{A} & \vm{B} \\ \vm{C} & \vm{D} \end{pmatrix}  
  \begin{pmatrix} 
 \vm{M} &  -\vm{M}\vm{B}\vm{D}^{-1}  \\
 -\vm{D}^{-1}\vm{C}\vm{M}  & \vm{D}^{-1} \vm{C}\vm{M}\vm{B}\vm{D}^{-1} 
 \end{pmatrix}
 = \begin{pmatrix} \vm{I} & \vm{O} \\ \vm{O} & \vm{I} \end{pmatrix}  \\
 \end{align*}
 \vspace{-1cm}
 \[\text{con } \vm{M} = (\vm{A}-\vm{B}\vm{D}^{-1}\vm{C})^{-1}  \]
\end{textblock}
}


\begin{textblock}{160}(0,64)
 \begin{align*}
\onslide<3->{  \vm{A}\vm{M} - \vm{B}\vm{D}^{-1}\vm{C}\vm{M}}
\onslide<4->{= \Big(\vm{A} - \vm{B}\vm{D}^{-1}\vm{C}\Big) \vm{M}} \onslide<5>{= \vm{M}^{-1}\vm{M}}
 \end{align*}
\end{textblock}


\end{frame}


\begin{frame}[plain]
 \begin{textblock}{160}(0,0)
\begin{center}
 \Large Distribuci\'on Gaussiana marginal
\end{center}
\end{textblock}
\vspace{0.75cm}

 \begin{textblock}{160}(0,12)
\begin{align*}
\onslide<1->{ p(\vm{x}_a) = \int p(\vm{x}_a,\vm{x}_b) d\vm{x}_b }
\end{align*}
\end{textblock}

\begin{textblock}{160}(0,24)
\begin{align*}
\only<2>{ -\frac{1}{2}\vm{x}_b^T \vm{\Lambda}_{bb}\vm{x}_b + \vm{x}_b^T \, (  \vm{\Lambda}_{bb}\bm{\mu}_b - \vm{\Lambda}_{ba}( \vm{x}_a - \bm{\mu}_a))  }
\only<3->{ -\frac{1}{2}\vm{x}_b^T \vm{\Lambda}_{bb}\vm{x}_b + \vm{x}_b^T \underbrace{(\vm{\Lambda}_{bb}\bm{\mu}_b - \vm{\Lambda}_{ba}(\vm{x}_a - \bm{\mu}_a))}_{\vm{m}} }
\end{align*}
\end{textblock}

\begin{textblock}{160}(0,38)
\begin{align*}
 \onslide<4->{ -\frac{1}{2}\vm{x}_b^T \vm{\Lambda}_{bb}\vm{x}_b + \vm{x}_b^T \vm{m} }
  \onslide<5->{ & =  -\frac{1}{2}\vm{x}_b^T \vm{\Lambda}_{bb}\vm{x}_b + \vm{x}_b^T \vm{m} \overbrace{- \frac{1}{2}\vm{m}^T \vm{\Lambda}_{bb}^{-1} \vm{m} + \frac{1}{2}\vm{m}^T \vm{\Lambda}_{bb}^{-1} \vm{m}}^{\vm{0}}  \\ }
  \onslide<6->{ & =   \underbrace{-\frac{1}{2} (\vm{x}_b - \vm{\Lambda}_{bb}^{-1}\vm{m})^T \vm{\Lambda}_{bb}(\vm{x}_b - \vm{\Lambda}_{bb}^{-1} \vm{m})}_{\text{Exponente de la Gaussiana}} +  \frac{1}{2}\vm{m}^T \vm{\Lambda}_{bb}^{-1} \vm{m} }
\end{align*}
\end{textblock}

\begin{textblock}{160}(0,70)
\begin{align*}
\onslide<7->{ p(\vm{x}_a) =  f(\vm{x}_a)  \int \exp\left( -\frac{1}{2} (\vm{x}_b - \vm{\Lambda}_{bb}^{-1}\vm{m})^T \vm{\Lambda}_{bb}(\vm{x}_b - \vm{\Lambda}_{bb}^{-1} \vm{m}) \right) d\vm{x}_b }
\end{align*}
\end{textblock}

\end{frame}


\begin{frame}[plain]
 \begin{textblock}{160}(0,0)
\begin{center}
 \Large Distribuci\'on Gaussiana marginal
\end{center}
\end{textblock}
\vspace{0.75cm}

\begin{textblock}{160}(0,18)
\begin{align*}
\onslide<1->{ f(\vm{x}_a) & = \frac{1}{2}\vm{m}^T \vm{\Lambda}_{bb}^{-1} \vm{m}  - \frac{1}{2}\vm{x}_a^T \vm{\Lambda}_{aa} \vm{x} + \vm{x}_a^T (\vm{\Lambda}_{aa} \bm{\mu}_a + \vm{\Lambda}_{ab} \bm{\mu}_b) + \text{const} \transparent{0} \text{cdddddonsssst}\\} 
\only<2>{ & \overset{*}{=} -\frac{1}{2} \vm{x}_a^T \, (\vm{\Lambda}_{aa}-\vm{\Lambda}_{ab}\vm{\Lambda}_{bb}^{-1}\vm{\Lambda}_{ba}) \, \vm{x}_a  + \vm{x}_a^T \, (\vm{\Lambda}_{aa}-\vm{\Lambda}_{ab}\vm{\Lambda}_{bb}^{-1}\vm{\Lambda}_{ba})^{-1} \bm{\mu}_a + \text{c}}
\only<3->{ & \overset{*}{=} -\frac{1}{2} \vm{x}_a^T \underbrace{ (\vm{\Lambda}_{aa}-\vm{\Lambda}_{ab}\vm{\Lambda}_{bb}^{-1}\vm{\Lambda}_{ba})}_{\vm{\Sigma}_a^{-1}} \vm{x}_a  + \vm{x}_a^T \underbrace{(\vm{\Lambda}_{aa}-\vm{\Lambda}_{ab}\vm{\Lambda}_{bb}^{-1}\vm{\Lambda}_{ba})^{-1} \bm{\mu}_a}_{\vm{\Sigma}_a^{-1}\bm{\mu}_a} +  \ \text{c}}
\end{align*}
\end{textblock}


\begin{textblock}{160}(0,46)
 \begin{align*}
  \only<4>{ \vm{\Sigma}_a =  (\vm{\Lambda}_{aa}-\vm{\Lambda}_{ab}\vm{\Lambda}_{bb}^{-1}\vm{\Lambda}_{ba})^{-1} \, \ \ \ \ \  \bm{\mu}_a = \vm{\Sigma}_a \vm{\Sigma}_a^{-1} \bm{\mu}_a  }
  \only<5->{ \vm{\Sigma}_a = \underbrace{(\vm{\Lambda}_{aa}-\vm{\Lambda}_{ab}\vm{\Lambda}_{bb}^{-1}\vm{\Lambda}_{ba})^{-1}}_{\vm{\Sigma}_{aa}} \ \ \ \ \  \bm{\mu}_a = \vm{\Sigma}_a \vm{\Sigma}_a^{-1} \bm{\mu}_a  }
 \end{align*}
\end{textblock}



\begin{textblock}{160}(0,66)
\onslide<6->{
\Wider[-8cm]{
\begin{mdframed}[backgroundcolor=black!20]
 \[  p(\vm{x}_a) = \N(\vm{x}_a | \bm{\mu}_a , \vm{\Sigma}_{aa}) \]
 \end{mdframed}
}
}
 \end{textblock}

\end{frame}



\begin{frame}[plain]
 \begin{textblock}{160}(0,0)
\begin{center}
 \Large Teorema de Bayes para Gaussiana multivariada
\end{center}
\end{textblock}
\vspace{0.75cm}

Dadas un modelo lineal
\begin{align*}
    p(\vm{w}) &=  \N(\vm{w}|\bm{\mu},\bm{\Lambda}^{-1}) \\
   p(\vm{y}|\vm{w}) &=  \N(\vm{y}|\vm{A}\vm{w}+\vm{b},\vm{L}^{-1})    
\end{align*}

\pause

Queremos probar,  
\begin{align*}
    p(\vm{y}) &=  \N(\vm{y}|\cdot,\cdot) \\ %\N(\vm{y}|\vm{A}\bm{\mu}+\vm{b},\, \vm{L}^{-1} + \vm{A}\vm{\Lambda}^{-1}\vm{A}^T) \\
   p(\vm{w}|\vm{y}) &=  \N(\vm{y}|\cdot,\cdot) %\N(\vm{w}|\vm{\Sigma}[\vm{A}^T \vm{L}(\vm{y}-\vm{b})+ \bm{\Lambda\mu}],\,  \vm{\Sigma})\label{eq:post}
\end{align*}

\pause 

Vamos a usar la probabilidad conjunta
\begin{align*}
 \vm{z} = \begin{pmatrix} \vm{w} \\ \vm{y}  \end{pmatrix}
\end{align*}

% 
% donde, 
% \begin{equation*}
%  \vm{\Sigma} = (\vm{\Lambda} + \vm{A}^T \vm{LA} )^{-1}
% \end{equation*}

\end{frame}


\begin{frame}[plain]
 \begin{textblock}{160}(0,0)
\begin{center}
 \Large Probabilidad conjunta \\
 \onslide<3->{\large T\'erminos cuadr\'aticos}
\end{center}
\end{textblock}

\begin{textblock}{160}(0,14)
\begin{align*}
 \onslide<1->{\log p(\vm{z}) &= \log p(w) + \log p(y|w) \\ }
 \onslide<2->{& = -\frac{1}{2}(\vm{w}-\bm{\mu})^T \vm{\Lambda}(\vm{w}-\bm{\mu}) -\frac{1}{2}(\vm{y}-\vm{A}\vm{w}-\vm{b})^T \vm{L}(\vm{w}-\vm{A}\vm{w}-\vm{b}) + \text{const} }
 \end{align*}
 \end{textblock}

 \begin{textblock}{140}(10,38)
 \onslide<4->{
 T\'erminos cuadr\'aticos}
 \begin{align*}
 \onslide<4->{
 &  -\frac{1}{2}\vm{w}^T(\vm{\Lambda} + \vm{A}^T\vm{L}\vm{A}) \vm{w}
   -\frac{1}{2}\vm{y}^T(\vm{L}) \vm{y}
   +\frac{1}{2}\vm{y}^T(\vm{L}\vm{A}) \vm{w}
   +\frac{1}{2}\vm{w}^T(\vm{A}^T\vm{L})\vm{y} 
   \\}
 \only<5>{
 & =  -\frac{1}{2} 
 \begin{pmatrix} \vm{w} \\ \vm{y}  \end{pmatrix}^T 
 \begin{pmatrix} \vm{\Lambda} + \vm{A}^T\vm{L}\vm{A}  &  -\vm{A}^T\vm{L}            \\ -\vm{L}\vm{A} & \vm{L}
  \end{pmatrix}
  \begin{pmatrix} \vm{w} \\ \vm{y}  \end{pmatrix}
 }
 \only<6->{
 & =  -\frac{1}{2} 
 \begin{pmatrix} \vm{w} \\ \vm{y}  \end{pmatrix}^T \underbrace{
 \begin{pmatrix} \vm{\Lambda} + \vm{A}^T\vm{L}\vm{A}  &  -\vm{A}^T\vm{L}            \\ -\vm{L}\vm{A} & \vm{L}
  \end{pmatrix}}_{\vm{R}}
  \begin{pmatrix} \vm{w} \\ \vm{y}  \end{pmatrix}
 }
 \end{align*}
 \end{textblock}

 
\begin{textblock}{160}(0,70)
 \begin{align*}
  \onslide<7->{\text{cov}[\vm{z}] = \vm{R}^{-1} = \begin{pmatrix} \vm{\Lambda}^{-1}   & \vm{\Lambda}^{-1}\vm{A}^T  \\
  \vm{A}\vm{\Lambda}^{-1} & \vm{L}^{-1} + \vm{A}\vm{\Lambda}^{-1}\vm{A}^T
  \end{pmatrix}}
 \end{align*}
\end{textblock}

\end{frame}

\begin{frame}[plain]
 \begin{textblock}{160}(0,0)
\begin{center}
 \Large Probabilidad conjunta \\
 \large T\'erminos lineales
\end{center}
\end{textblock}

\begin{textblock}{160}(0,14)
\begin{align*}
 \onslide<1->{\log p(\vm{z}) &= \log p(w) + \log p(y|w) \\ }
 \onslide<1->{& = -\frac{1}{2}(\vm{w}-\bm{\mu})^T \vm{\Lambda}(\vm{w}-\bm{\mu}) -\frac{1}{2}(\vm{y}-\vm{A}\vm{w}-\vm{b})^T \vm{L}(\vm{w}-\vm{A}\vm{w}-\vm{b}) + \text{const} }
 \end{align*}
 \end{textblock}

 \begin{textblock}{140}(10,38)
 \onslide<2->{
 T\'erminos lineales}
 \begin{align*}
 \onslide<2->{\vm{w}^T\vm{\Lambda}\bm{\mu}-\vm{w}^T\vm{A}^T\vm{L}\vm{b} + \vm{y}^T \vm{L}\vm{b}  = \begin{pmatrix} \vm{w} \\ \vm{y}  \end{pmatrix}^T \begin{pmatrix} \vm{\Lambda}\bm{\mu} - \vm{A}^T\vm{L}\vm{b} \\ \vm{L}\vm{b} \end{pmatrix} }
 \end{align*}
 \end{textblock}

 \begin{textblock}{160}(0,60)
 \begin{align*}
  \onslide<3->{\text{E}[\vm{z}] = \vm{R}^{-1} \begin{pmatrix} \vm{\Lambda}\bm{\mu} - \vm{A}^T\vm{L}\vm{b} \\ \vm{L}\vm{b} \end{pmatrix} } 
  \onslide<4->{ = \begin{pmatrix} \bm{\mu} \\ \vm{A}\bm{\mu}+\vm{b} \end{pmatrix} }
 \end{align*}
 \end{textblock}

 \end{frame}


\begin{frame}[plain]
 \begin{textblock}{160}(0,0)
\begin{center}
 \Large Evidencia y Posterior \\
\end{center}
\end{textblock}
\vspace{1cm}

Dado: \ \ \ \ \ $ p(\vm{w}) =  \N(\vm{w}|\bm{\mu},\bm{\Lambda}^{-1}) \ \ \ \ \ \
   p(\vm{y}|\vm{w}) =  \N(\vm{y}|\vm{A}\vm{w}+\vm{b},\vm{L}^{-1})    $
   \vspace{0.25cm}
   
\Wider[2cm]{
\begin{align*}
 \vm{z} = \begin{pmatrix} \vm{w} \\ \vm{y}  \end{pmatrix}  \ \  \text{E}_{\vm{Z}} = \begin{pmatrix} \bm{\mu} \\ \vm{A}\bm{\mu}+\vm{b} \end{pmatrix} \ \ \text{cov}_{\vm{Z}} = \begin{pmatrix} \vm{\Lambda}^{-1}   & \vm{\Lambda}^{-1}\vm{A}^T  \\
  \vm{A}\vm{\Lambda}^{-1} & \vm{L}^{-1} + \vm{A}\vm{\Lambda}^{-1}\vm{A}^T  \end{pmatrix}  \ \ \text{cov}_{\vm{Z}}^{-1} = \begin{pmatrix} \vm{\Lambda} + \vm{A}^T\vm{L}\vm{A}  &  -\vm{A}^T\vm{L}            \\ -\vm{L}\vm{A} & \vm{L}
  \end{pmatrix}
\end{align*}
}
\pause1
\vspace{1cm}

Luego, usando la propiedades de la marginal y la condicional,
\pause
\begin{align*}
 p(\vm{y}) &= \N(\vm{y}| \vm{A}\bm{\mu} + \vm{b} , \ \   \vm{L}^{-1} + \vm{A}\vm{\Lambda}^{-1}\vm{A}^T ) \\[0.3cm]
 p(\vm{w}|\vm{y}) &= \N(\vm{w} | \vm{\Sigma}\left( \vm{A}^T \vm{L} (\vm{y}-\vm{b}) + \vm{\Lambda} \bm{\mu}  \right), \ \ \vm{\Sigma}) \\ 
 &  \text{ con } \vm{\Sigma} = (\vm{\Lambda} + \vm{A}^T \vm{L}\vm{A})^{-1}
\end{align*}

\end{frame}
 
\begin{frame}[plain]
 \begin{textblock}{100}(30,0)
 \begin{figure}
\begin{subfigure}[t]{0.32\textwidth} 
\caption*{Verosimilitud} 
\end{subfigure}
\begin{subfigure}[t]{0.32\textwidth}
\caption*{Priori/Posteriori} 
\includegraphics[width=\textwidth]{../../figures/linearRegression_posterior_0.png} 
\end{subfigure}
\begin{subfigure}[t]{0.32\textwidth}
\caption*{Data space} 
\includegraphics[width=\textwidth]{../../figures/linearRegression_dataSpace_0.png} 
\end{subfigure}


\begin{subfigure}[c]{0.32\textwidth}
\includegraphics[width=\textwidth]{../../figures/linearRegression_likelihood_1.png} 
\end{subfigure}
\begin{subfigure}[c]{0.32\textwidth}
\includegraphics[width=\textwidth]{../../figures/linearRegression_posterior_1.png} 
\end{subfigure}
\begin{subfigure}[c]{0.32\textwidth}
\includegraphics[width=\textwidth]{../../figures/linearRegression_dataSpace_1.png} 
\end{subfigure}

\begin{subfigure}[c]{0.32\textwidth}
\includegraphics[width=\textwidth]{../../figures/linearRegression_likelihood_2.png} 
\end{subfigure}
\begin{subfigure}[c]{0.32\textwidth}
\includegraphics[width=\textwidth]{../../figures/linearRegression_posterior_2.png} 
\end{subfigure}
\begin{subfigure}[c]{0.32\textwidth}
\includegraphics[width=\textwidth]{../../figures/linearRegression_dataSpace_2.png} 
\end{subfigure}

\end{figure}
\end{textblock}

\end{frame}

\begin{frame}[plain]
\begin{textblock}{140}(10,0)
 \begin{center}
  \Large Selecci\'on de modelo
 \end{center}
\end{textblock}

\begin{textblock}{80}(0,20)
 \centering
Funci\'on objetivo
\end{textblock}

\begin{textblock}{80}(80,20)
 \centering
 Modelos polinomiales
\end{textblock}

\begin{textblock}{140}(10,28)
     \centering 
       \includegraphics[width=0.48\textwidth]{../../figures/model_selection_true_and_sample.png} 
       \includegraphics[width=0.48\textwidth]{../../figures/model_selection_MAP.png} 
\end{textblock}

\end{frame}


\begin{frame}[plain]
\begin{textblock}{160}(0,0)
\begin{center}
 \Large Selecci\'on de modelo \\
 \large Evidencia
\end{center}
\end{textblock}


\only<1>{
\begin{textblock}{160}(0,24) 
\begin{equation*}
 P(\text{Modelo}|\text{Datos}) = \frac{P(\text{Datos}|\text{Modelo})P(\text{Modelo})}{ P(\text{Datos})}
\end{equation*}
\end{textblock}
}

\only<2>{
\begin{textblock}{160}(0,24)
\begin{equation*}
   \frac{P(\text{Modelo}_i|\text{Datos})}{P(\text{Modelo}_j|\text{Datos})}  = \frac{P(\text{Datos}|\text{Modelo}_i)\,\,P(\text{Modelo}_i)}{P(\text{Datos}|\text{Modelo}_j)\,\,P(\text{Modelo}_j)}  
\end{equation*}
\end{textblock}
}
\only<3>{
\begin{textblock}{160}(0,24)
\begin{equation*}
 \frac{P(\text{Modelo}_i|\text{Datos})}{P(\text{Modelo}_j|\text{Datos})}  = \frac{P(\text{Datos}|\text{Modelo}_i)\,\,P(\text{Modelo}_i)}{ \underbrace{P(\text{Datos}|\text{Modelo}_j)}_{\text{Evidencia!}}\,P(\text{Modelo}_j)}
\end{equation*}
\end{textblock}
}

\only<4->{
\begin{textblock}{160}(0,24)
\begin{equation*}
  \frac{p(\text{Modelo}_i|\text{Datos})}{p(\text{Modelo}_j|\text{Datos})}  = \frac{p(\text{Datos}|\text{Modelo}_i)\,\,p(\text{Modelo}_i)}{ \underbrace{p(\text{Datos}|\text{Modelo}_j)}_{\text{Evidencia!}}\, \underbrace{p(\text{Modelo}_j)}_{\text{Escalar } k } }
\end{equation*}
\end{textblock}
}


\begin{textblock}{160}(0,40)
\begin{equation*}
\begin{split}
\only<5->{p(\text{Datos}|\text{Modelo}) &= \sum_{h}^H p(\text{Datos}|\text{Hip\'otesis}=h,\text{Modelo}) p(\text{Hip\'otesis}=h|\text{Modelo}) } \\
\only<6->{& = p(\text{D}_1|\text{M}) p(\text{D}_2|\text{D}_1, \text{M}) \dots p(\text{D}_N|\text{D}_{N-1} \dots \text{D}_1, \text{M}) }
\end{split} 
\end{equation*}
\end{textblock}

\only<7>{
\begin{textblock}{140}(10,68)
\begin{mdframed}[backgroundcolor=black!15]
\footnotesize \centering Preferimos modelos con la \textbf{menor sorpresa conjunta}!
\end{mdframed}
 \end{textblock} 
}
 
\end{frame}


\begin{frame}[plain]
\begin{textblock}{160}(0,0)
\begin{center}
 \large Evidencia vs M\'aximo A Posteriori \\
 \normalsize offline
\end{center}
\end{textblock}



\begin{textblock}{80}(0,14)
 \centering \footnotesize
 \begin{align*}
 p(\text{Datos}|\text{Modelo})
\end{align*}
Evidencia conjunta
\end{textblock}

\begin{textblock}{80}(80,14)
 \centering \footnotesize
\begin{align*}
   p(\text{Datos}| h_{\text{MAP}}, \text{Modelo}) 
\end{align*}
 M\'aximo A Posteriori \\ \scriptsize
 (Likelihood regularizado)
\end{textblock}


\begin{textblock}{140}(10,28)
     \centering 
       \begin{figure}[H]     
     \centering 
     \begin{subfigure}[b]{0.47\textwidth}
       \includegraphics[width=1\textwidth]{../../figures/model_selection_evidence.png}
     \end{subfigure}
     \begin{subfigure}[b]{0.49\textwidth}
       \includegraphics[width=1\textwidth]{../../figures/model_selection_maxLikelihood.png}
     \end{subfigure}
\end{figure}
\end{textblock}


\end{frame}


\begin{frame}[plain]
\begin{textblock}{160}(0,0)
\begin{center}
 \large Evidencia vs M\'aximo A Posteriori \\
 \normalsize online
\end{center}
\end{textblock}


\begin{textblock}{60}(0,14)
 \centering \footnotesize
 \begin{align*} 
 p(\text{D}_1|\text{M}) \dots p(\text{D}_N|\text{D}_{N-1} \dots \text{D}_1, \text{M})
\end{align*}
Evidencia conjunta
\end{textblock}

\begin{textblock}{80}(60,14)
 \centering \footnotesize
\begin{align*}
 p(\text{D}_1|h_0 , \text{M}) p(\text{D}_2|h_{\text{MAP}_1}, \text{D}_1, \text{M}) \dots p(\text{D}_N|h_{\text{MAP}_{N-1}} , \text{D}_{N-1} \dots \text{D}_1, \text{M})
\end{align*}
 M\'aximo A Posteriori
\end{textblock}


\begin{textblock}{140}(10,28)
     \centering 
       \begin{figure}[H]     
     \centering 
     \begin{subfigure}[b]{0.47\textwidth}
       \includegraphics[width=1\textwidth]{../../figures/model_selection_evidence.png}
     \end{subfigure}
     \begin{subfigure}[b]{0.49\textwidth}
       \includegraphics[width=1\textwidth]{../../figures/model_selection_maxApriori_online.png}
     \end{subfigure}
\end{figure}
\end{textblock}



\end{frame}


\begin{frame}[plain]
\begin{textblock}{140}(10,0)
 \begin{center}
  \Large  Evidencia
 \end{center}
\end{textblock}


 \begin{textblock}{130}(10,14)
  \centering
  \includegraphics[width=0.8\textwidth]{../../figures/evidencia_de_modelos_alternativos} 
 \end{textblock} 
 
 
 \begin{textblock}{120}(20,76)
  \begin{mdframed}[backgroundcolor=black!15]
\centering
  Balance natural entre complejidad y predicci\'on
  \end{mdframed}
 \end{textblock}
\end{frame}
 
\begin{frame}[plain]
\centering
  \includegraphics[width=0.35\textwidth]{../../../aux/static/pachacuteckoricancha.jpg}
\end{frame}






\end{document}



