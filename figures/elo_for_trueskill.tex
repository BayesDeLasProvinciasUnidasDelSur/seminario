\documentclass[10pt]{standalone}
\usepackage[utf8]{inputenc}
\usepackage{tikz}

\input{../../../aux/tikzlibrarybayesnet.code.tex}
\input{../../../aux/encabezado.tex}
\makeatletter
\newcommand{\vast}{\bBigg@{2.5}}
\newcommand{\Vast}{\bBigg@{14.5}}
\makeatother

\usepackage{helvet}
\renewcommand{\familydefault}{\sfdefault}
\begin{document}

\tikz{ %
        
        \node[det, fill=black!10] (r) {$r_{ij}$} ; %
        \node[const, left=of r, xshift=-1.4cm] (r_name) {\small Resultado observado ($r$)}; 
        \node[const, right=of r] (dr) {\large $ r_{ij} = \mathbb{I}(p_i>p_j)$}; 
          
         \node[latent, above=of r, xshift=-0.8cm] (p1) {$p_i$} ; %
         \node[latent, above=of r, xshift=0.8cm] (p2) {$p_j$} ; %
         \node[const, left=of p1, xshift=-0.55cm, yshift=0.2cm] (p_name) {\small Rendimiento aleatorio oculto ($p$)}; 
         \node[const, left=of p1, xshift=-0.55cm, yshift=-0.2cm] (p_name) {\small centrado en la habilidad estimada ($s$)}; 
         
         \node[latent, above=of p1,color=white] (s1) {$s_i$} ; %
         \node[latent, above=of p2, color=white] (s2) {$s_j$} ; %
                  
         \node[const, right=of p2] (dp2) {\large $p \sim N(s,\beta^2)$};
         
%          \node[latent, above=of p1, fill=black, minimum size=1pt] (s1) {} ; %
%          \node[latent, above=of p2, fill=black, minimum size=1pt] (s2) {} ; %
%          \node[const, above=of s1] (ds1) {\large $s_i$};
%          \node[const, above=of s2] (ds2) {\large $s_j$};
%          \node[const, left=of s1, xshift=-.85cm, yshift=0.2cm] (s_name) {Habilidad estimada}; 
%          
         \edge {p1,p2} {r};
         %\edge {s1} {p1};
         %\edge {s2} {p2};
         
         \node[invisible, right=of p2, xshift=4.75cm] (s-dist) {};
} 


\end{document}
